\section{Sources of Radiation}
This chapter is a quick review of NE101 content. Topics such as Q-value calculation, various types of decay mechanisms, and decay calculation are covered. In these notes, only a quick summary will be provided.
\subsection{Various Ionizing Radiation and Spectra}
\begin{itemize}
    \item Fast electrons
    \begin{itemize}
        \item Beta decay: continuous spectrum\\
        Note that electron capture creates a discrete spectrum
        \item Internal conversion: discrete spectrum\\
        Note that these electrons are shell electrons, so energies are lower at $\approx$100's of keV
        \item Auger electrons: discrete spectrum\\
        These are also shell electrons, energies with unit of keV
        \item Secondary electrons: discrete or continuous spectrum depending on type\\
        Compton electrons (continuous), photo-electrons (discrete), or delta rays (continuous)\\
        Delta rays are created when some heavy-charged particle knocks an electron out of its orbital in an atom, and these electrons have enough energy to cause further ionization. 
        \item Accelerator based: discrete spectrum
    \end{itemize}
    \item Heavy charged particles
    \begin{itemize}
        \item Alpha decay: discrete spectrum
        \item Spontaneous fission (fission fragments): continuous 
    \end{itemize}
    \item Photons
    \begin{itemize}
        \item Gamma decay: discrete spectrum
        \item Annihilation: ''discrete`` spectrum ($\approx$511 keV)
        \item Characteristic X-rays: discrete spectrum  
        \item Bremsstralung: continuous spectrum\\
        Bremsstralung radiation is created when fast electrons interact with matter
    \end{itemize}
    \item Neutrons
    \begin{itemize}
        \item Spontaneous Fission: continuous spectrum
        \item ($\alpha$,n) sources: continuous spectrum!!!\\
        Note that while the $\alpha+ \;^9\text{Be}$ reaction has a set Q-value, becuase the $\alpha$ particle loses a varying amount of energy before reaching the Be-9 nucleus, the spectrum is continuous.
        \item ($\gamma$,n) sources: nearly discrete\\
        Since the gamma rays are mono-energetic (contrast with aforementioned alpha particles), the neutrons produced are also nearly mono-energetic. The small energy spread comes from the varying angle between the incoming gamma direction and outgoing neutron direction. 
        \item Neutron generators: nearly discrete
    \end{itemize}
\end{itemize}
