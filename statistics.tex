\section{Uncertainty Analysis, Counting Statistics, and Detection}
\subsection{Uncertainty Analysis}
\subsubsection{Characterization of Data}
With $N$ independent measurements of the same physical quantity:
\begin{itemize}
    \item Experimental mean: $\Bar{x_e}=\frac{1}{N}\sum_{i=1}^Nx_i$
    \item Residual: $d_i=x_i-\Bar{x_e},\text{ with }\sum_{i=1}^Nd_i=0$
    \item Deviation: $\epsilon_i=x_i-\Bar{x}$, note that $\Bar{x}$ is the true mean 
    \item Sample variance: $s^2=\Bar{\epsilon^2}=\frac{1}{N}\sum_{i=1}^N(x_i-\Bar{x})^2=\frac{1}{N-1}\sum_{i=1}^N(x_i-\Bar{x_e})^2$
    \item Standard Deviation: $\sigma^2=\frac{1}{N}\sum_{i=1}^N(x_i-\Bar{x})^2=\frac{1}{N-1}\sum_{i=1}^N(x_i-\Bar{x_e})^2$, or $\sigma^2\approx\Bar{x_i^2}-(\Bar{x_e})^2$\\
    The standard deviation is also known as the external uncertainty in the mean.
\end{itemize}
\subsubsection{Types of Uncertainties}
\begin{itemize}
    \item Internal Uncertainty
    \item External Uncertainty
\end{itemize}
\subsubsection{Error Propagation Rules}
\begin{itemize}
    \item Assume $N$ independent measurements of the same physical quantity, to which $F$ is related by $F=F(x_1,x_2,...)$:\\
    then the uncertainty in F is given by $\sigma_F^2=\sum_{i=1}^N\left(\frac{\partial F}{\partial x_i}\right)^2\sigma^2_{x_i}$
    \item Examples
    \begin{itemize}
        \item Sum and differences: $\sigma_{x+y}=\sigma_{x-y}=\sqrt{\sigma_x^2+\sigma_y^2}$
        \item Multiplication and division by constant: $\sigma_{Kx}=K\sigma_x$
        \item Multiplication: $\sigma_{x\cdot y}=(x\cdot y)\sqrt{\left(\sigma_x/x\right)^2+\left(\sigma_y/y\right)^2}$
        \item Division: $\sigma_{x/y}=({x}/{y})\sqrt{\left(\sigma_x/x\right)^2+\left(\sigma_y/y\right)^2}$
    \end{itemize}
    \item Counting Rate ($R$)
    \begin{itemize}
        \item $R=n/T$ and $\sigma_n=\sqrt{n}\;\Rightarrow\;\sigma_R=\sqrt{n}/{T}=\sqrt{R/T}$
        \item $R_S=R_{S+B}-R_B\;\Rightarrow\;\sigma_{R_S}=\sqrt{\frac{n_{S+B}}{T_{S+B}^2}+\frac{n_{B}}{T_{B}^2}}=\sqrt{\frac{R_{S+B}}{T_{S+B}}+\frac{R_B}{T_B}}$
    \end{itemize}
\end{itemize}
\subsubsection{Rounding of Values}
Rounding to 1-2 significant figures in the uncertainty, demonstrated by the following examples:
\begin{center}
\begin{tabular}{c c c}
    Raw value & $n=1$ & $n=2$\\
    \hline
    $5.73297251\pm0.01477602$ &  $5.73\pm0.01$ & $5.733\pm0.015$ \\
    $314775089\pm4500284$ & $(3.15\pm0.05)\times10^8$ & $(3.148\pm0.045)\times10^8$ \\
    $255\pm73$ & $(2.6\pm0.7)\times10^2$ & $255\pm73$
\end{tabular}  
\end{center}
\subsection{Counting Statistics}
\subsubsection{Statistical Models}

\subsubsection{Confidence Limits}
\subsubsection{The \texorpdfstring{$\chi^2$}{Chi-squared} Test}
\subsubsection{Examples}
\subsection{Detection}
\subsubsection{Dead Time}
\subsubsection{Decision Making}