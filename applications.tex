\section{Applications}
\subsection{Gamma Ray Concepts and Usages}
\begin{itemize}
    \item $\gamma$-rays (X-rays) are fingerprints of nuclei (atoms)
    \item The can be used in the following modes:
    \begin{itemize}
        \item Emission:\\
        Emission of gamma rays from the source provides information on the source itself, its environment, transport, and distribution.
        \item Induced emission:\\
        Provides information similar to above, but in examples such as NAA or XRF (see later sections).
        \item (Induced) Transmission or scattering:\\
        Transmission or scattering of photons provides information on the transmitting/ scattering object.
    \end{itemize}
\end{itemize}
\subsection{Carbon Dating}
\subsubsection{Assumptions and Concepts}
\begin{itemize}
    \item It is assumed that
    \begin{itemize}
        \item The rate of C-14 production is constant over time ($t\gg t_{1/2}$).
        \item The C-14 to C-12 ratio was in the same equilibrium as in the organisms today.
        \item The half-life of C-14 is known (measured to be 5730(40) years). Note: According to NNDC, the half-life is 5700(30) years.
    \end{itemize}
    \item C-14 is formed through $^{14}$N(n,p) from cosmic-ray produced neutrons.
    \item C-14 enters living organisms as CO$_2$, and can be found in all living organism with a constant C-14 to C-total ratio.
    \item The death of the organism sets the clock, and C-14 is depleted via beta decay.
\end{itemize}
\subsubsection{Measurement}
\textcolor{red}{CHECK THIS SECTION!!!!!}
\begin{itemize}
    \item With the decay law,
    \begin{itemize}
        \item[] $N=N_0\exp(-\lambda t)$
        \item[] $\Rightarrow\;t=-\ln(N/N_0)/\lambda$
        \item[] $\Rightarrow\;t=\ln(C_0/C)\times t_{1/2}/\ln(2)$
        \item[] , where $C_0$ is the C-14 to C-total ratio when the organism died, and $C$ is the ratio measured in the sample. $C_0$ is assumed to be same as current living organisms.
    \end{itemize}
\end{itemize}
\subsubsection{Limitations}
\begin{itemize}
    \item The C-14 in samples have low specific activity:
    \begin{itemize}
        \item Low count rates even in living organsism
        \item Limited time range of $<50,000$ years (accelerator-based atomic mass spectrometry provides higher specificity, enabling ranges up to 100,000 years)
    \end{itemize}
    \item C-14 concentration changes over time:
    \begin{itemize}
        \item Global and local changes in cosmic ray flow and climate, etc. (e.g., increased use of fossil fuel and atmospheric nuclear testing)
        \item Independent calibration required for C-14 concentration
    \end{itemize}
    \item Method only sensitive to carbon-based living organisms
    \item Other radiometric dating techniques include K-Ar dating and U-Pb dating (both with half-life on the scale of $10^{9}$ years).
\end{itemize}
\subsection{Neutron Activation Analysis and X-Ray Fluorescence}
\subsubsection{Neutron Activation Analysis}
\begin{itemize}
    \item Coupled with gamma spectrometry, commonly used due to ease and low cost for determination of trace impurities ($\sim10^{-12}$) 
    \item Minimum-invasive, independent of chemical form
    \item Sensitivity varies due to differences in: 
    \begin{itemize}
        \item Neutron cross sections
        \item Natural isotopic abundances
        \item Half-lives
        \item Gamma-ray branching
    \end{itemize}
    \item Main constituents of living tissue (H, O, Si, P, S, Pb, Bi) cannot be detected
\end{itemize}
\subsubsection{X-Ray Fluorescence}
\begin{itemize}
    \item X-rays, $\gamma$-rays, or charged particles are used to excite atoms for emission of K or L shell X-rays.
    \item High resolution Si detectors enable XRF down to boron.
    \item Advantages over NAA include: small, self-contained units and no reactor irradiation required
    \item Can reach comparable relative sensitives as in NAA
\end{itemize}
\subsection{Gamma-Ray Spectroscopy and Imaging}
\subsubsection{Gamma-Ray Tracking}
\subsubsection{Gamma-Ray Imaging}